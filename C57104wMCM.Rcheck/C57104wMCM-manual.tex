\nonstopmode{}
\documentclass[a4paper]{book}
\usepackage[times,inconsolata,hyper]{Rd}
\usepackage{makeidx}
\makeatletter\@ifl@t@r\fmtversion{2018/04/01}{}{\usepackage[utf8]{inputenc}}\makeatother
% \usepackage{graphicx} % @USE GRAPHICX@
\makeindex{}
\begin{document}
\chapter*{}
\begin{center}
{\textbf{\huge Package `C57104wMCM'}}
\par\bigskip{\large \today}
\end{center}
\ifthenelse{\boolean{Rd@use@hyper}}{\hypersetup{pdftitle = {C57104wMCM: Simplified IEEE C57.104-2019 DGA Oil-Immersed Transformer Screening Methodology With Integrated Monte Carlo Propagated Uncertainty}}}{}
\begin{description}
\raggedright{}
\item[Type]\AsIs{Package}
\item[Title]\AsIs{Simplified IEEE C57.104-2019 DGA Oil-Immersed Transformer
Screening Methodology With Integrated Monte Carlo Propagated
Uncertainty}
\item[Version]\AsIs{0.1.0}
\item[Author]\AsIs{Michael Hosseini}
\item[Maintainer]\AsIs{Michael Hosseini }\email{michaelhosseini3141@gmail.com}\AsIs{}
\item[Description]\AsIs{Applies a simplified version of the IEEE C57.104-2019 DGA 
Screening Methodology with added uncertainty propagation via a Monte Carlo
Method (MCM). It will output the per-gas DGA Status and Combined DGA Status 
as well as an estimated probability of each given outcomes based on the 
MCM. There is an optional Diagnostic step that will apply a combination of
Duval Triangles 1|4|5 (C57.104 Tables 6, D.3, and D.4, respectively), 
Rogers Ratio (C57.104 Table 5), and IEC 60599(2022) Tables 1|2 to the 
samples. It will provide diagnostics split in accordance to the Screening 
outputs. As a reminder, the IEEE Screening Methodology generally recommends
diagnostics only for a DGA Status of 2 or 3. The function will attempt to
use all gas samples provided; i.e., not limit Table 4 to 3-6 samples. A 
Normal, Triangular, or Uniform distribution shape may be selected. If the 
first, the uncertainty inputted is assumed to represent a 95% probability 
level. If the latter two, then the uncertainty is assumed to represent a 
100% probability level. Accuracy is interpretted  as a 0-1 %-equivalent of
the gas sample value whereas the minimum accuracy is interpretted as 
absolute values in PPM. If desired, a correlation can be added across gases
(i.e. within a given sample). This can be either as a rho or a correlation 
matrix. Please refer to my thesis as to some of the limitations, namely the
omission of correlations across samples to accommodate the Accuracy and 
Reproducibility as defined by ISO 5725 + IEC 60567.}
\item[License]\AsIs{GPL-3}
\item[Encoding]\AsIs{UTF-8}
\item[Imports]\AsIs{roxygen2}
\item[Roxygen]\AsIs{list(markdown = TRUE)}
\item[RoxygenNote]\AsIs{7.3.2}
\item[NeedsCompilation]\AsIs{no}
\end{description}
\Rdcontents{Contents}
\HeaderA{C57104wMCM}{C57104wMCM: Simplified C57.104 DGA Screening with propagated uncertainty via a Monte Carlo Method}{C57104wMCM}
%
\begin{Description}
This function applies a simplified version of the C57.104 DGA Screening methodology with propagated uncertainty. This is achieved by generating random samples according to the statistical description. These outcomes for these random samples are used as proxies for estimating the likelihood of different outcomes. There is also an integrated diagnostic step available using a combination of Duval Triangles 1|4|5, Rogers Ratio Method and IEC 60599:2022 Tables 1|2.
\end{Description}
%
\begin{Usage}
\begin{verbatim}
C57104wMCM(
  gas_data,
  limit_data,
  accuracy,
  min_accuracy,
  distribution_shape = "n",
  rho = 0,
  diagnosis = 0,
  n = 1000,
  flex = FALSE,
  silent = FALSE,
  verbose = FALSE,
  validate = FALSE
)
\end{verbatim}
\end{Usage}
%
\begin{Arguments}
\begin{ldescription}
\item[\code{gas\_data}] Data.frame: First column containing date of sample. Cannot have two samples on same day. Rest of columns containing sample value in PPM.

\item[\code{limit\_data}] Data.frame: Columns containing the IEEE DGA Table limits for each gas. Each Table is on its own row; four rows expected.

\item[\code{accuracy}] Named Numeric: Values to be multiplied by gas value to give uncertainty. Can either be same value for all gases, or individual values for each gas. Expected range: 0-1.

\item[\code{min\_accuracy}] Named Numeric: Absolute Values (PPM) used as minimum uncertainty. Can either be same value for all gases, or individual values for each gas.

\item[\code{distribution\_shape}] Numeric: Distribution shape representing uncertainty. "N" = Normal, "T" = Triangular, "U" = Uniform. If Normal, Accuracy is interpreted as 95\% confidence interval, else as 100\%.

\item[\code{rho}] Numeric: Value representing rho for correlations. Can either be same value for all gases, individual values for each gas, or a matrix intended to represent the \code{cor\_mat}.

\item[\code{diagnosis}] Mixed: Used to specify which diagnoses to conduct. 0 = None, 1 = All, "DT" = Duval Triangles 1|4|5 only, "RR" = Rogers Ratio Only, "IEC" = IEC Tables 1|2 Only.

\item[\code{n}] Numeric: Number of samples to generate for the Monte Carlo for each gas. If a value <1, it is assumed as an estimate for the error tolerance, then Wald Interval used to estimate n.

\item[\code{flex}] Boolean: Defaulted to FALSE. If TRUE, and \code{rho} is assumed a correlation matrix equivalent, it will attempt to force it to become a positive definite if needed.

\item[\code{silent}] Boolean: Defaulted to FALSE. If TRUE: heavily restricts outputs to console.

\item[\code{verbose}] Boolean: Defaulted to FALSE. If FALSE: restricts outputs to console.

\item[\code{validate}] Boolean: Defaulted to FALSE. If TRUE: provides various validatory plots.
\end{ldescription}
\end{Arguments}
%
\begin{Value}
Nested List containing:
\begin{itemize}

\item{} \$\code{inputs}:
\begin{itemize}

\item{} \$\code{gas\_data}: Data.frame. First column as Date converted to days since oldest sample. Rest as gas values in PPM.
\item{} \$\code{limit\_data}: Data.frame. Unchanged from input.
\item{} \$\code{accuracy}: Numeric. Unchanged from input.
\item{} \$\code{min\_accuracy}: Numeric. Unchanged from input.
\item{} \$\code{distribution\_shape}: Numeric. 1 = Normal, 2 = Triangular, 3 = Uniform.
\item{} \$\code{rho}: Numeric or NA. NA if \code{cor\_mat} is being used.
\item{} \$\code{diagnosis}: Numeric. 0 = None, 1 = All, 2 = Duval Triangles, 3 = Rogers Ratios, 4 = IEC Tables.
\item{} \$\code{n}: Numeric. Number of samples used for Monte Carlo Simulation.
\item{} \$\code{col\_names}: Names of found gas names.
\item{} \$\code{cor\_mat}: Matrix or NA. NA if \code{rho} is being used.
\item{} \$\code{sd\_data}: Data.frame. Same as \code{gas\_data} but with the values representing Std Dev and not gas values.

\end{itemize}

\item{} \$\code{sampled}: Optional: (\code{validate} == TRUE): Numeric Array. Values representing generated samples for the Monte Carlo based on the statistical description provided. Shaped: Days x Gases x Samples.
\item{} \$\code{screened}:
\begin{itemize}

\item{} \$\code{protocol}: Numeric. 1|2|3 depending on if 1, 2, or more samples.
\item{} \$\code{T<n>\_default}: Output for Tables 1|2|3|4 using default gas values.
\item{} \$\code{T<n>}: Probability of output for Tables 1|2|3|4 using generated samples.
\item{} \$\code{T<n>\_bounds}: Numeric. Associated estimate of confidence interval.
\item{} \$\code{gas\_L<n>\_default}: Numeric. Output for per-gas DGA Status Levels 1|2|3.
\item{} \$\code{gas\_L<n>}: Numeric. Probability of output for per-gas DGA Status Levels 1|2|3.
\item{} \$\code{gas\_L<n>\_bounds}: Numeric. Associated confidence interval.
\item{} \$\code{L<n>\_default}: Numeric. Output for combined DGA Status Levels 1|2|3.
\item{} \$\code{L<n>}: Numeric. Probability of output for combined DGA Status Levels 1|2|3.
\item{} \$\code{L<n>\_bounds}: Numeric. Mask indicated which generated samples were combined DGA Status Levels 1|2|3.
\item{} \$\code{L<n>\_mask}: Optional: (\code{validate} == TRUE): Logical Array. Associated estimate of confidence interval.
\item{} \$\code{T12\_vals\_default}: Optional: (\code{validate} == TRUE): Numeric. Metric values using default gas values used for Tables 1 and 2.
\item{} \$\code{T3\_vals\_default}: Optional: (\code{validate} == TRUE): Numeric. Metric values using default gas values used for Table 3.
\item{} \$\code{T4\_vals\_default}: Optional: (\code{validate} == TRUE): Numeric. Metric values using default gas values used for Tables 4.
\item{} \$\code{T12\_vals}: Optional: (\code{validate} == TRUE): Numeric. Metric values using generated samples used for Tables 1 and 2.
\item{} \$\code{T3\_vals}: Optional: (\code{validate} == TRUE): Numeric. Metric values using generated samples used for Table 3.
\item{} \$\code{T4\_vals}: Optional: (\code{validate} == TRUE): Numeric. Metric values using generated samples used for Tables 4.
\item{} \$\code{GL<n>\_mask}: Optional: (\code{validate} == TRUE): Logical Array. Mask indicated which generated samples were per-gas DGA Status Levels 1|2|3.

\end{itemize}

\item{} \$\code{diagnoses}: Optional: (\code{diagnosis} != 0):
\begin{itemize}

\item{} \$\code{DT<n>\_default\_data}: Optional: (\code{diagnosis} == 1|2): Numeric. Ratios relevant to DT 1|4|5 using default gas values.
\item{} \$\code{DT<n>\_0\_data}: Optional: (\code{diagnosis} == 1|2): Numeric Array. Ratios relevant to DT 1|4|5 using generated samples.
\item{} \$\code{DT<n>\_L<n>}: Optional: (\code{diagnosis} == 1|2): Numeric. Proportion of samples, screened at DGA Status 1|2|3, assigned given diagnosis in accordance to DT 1|4|5.
\item{} \$\code{DT<n>\_Ln1}: Optional: (\code{diagnosis} == 1|2): Numeric. Proportion of samples, screened at DGA Status 2 or 3, assigned given diagnosis in accordance to DT 1|4|5.
\item{} \$\code{gas\_L<n>\_default}: Optional: (\code{diagnosis} == 1|3|4): Numeric. Output for per-gas DGA Status Levels 1|2|3.
\item{} \$\code{Abs\_default\_default}: Optional: (\code{diagnosis} == 1|3|4): Numeric. Ratios relevant to RR and/or IEC 1|2 using default gas values.
\item{} \$\code{Abs\_0\_data}: Optional: (\code{diagnosis} == 1|3|4): Numeric Array. Ratios relevant to RR and/or IEC 1|2 using generated samples.
\item{} \$\code{RR\_L<n>}: Optional: (\code{diagnosis} == 1|3): Numeric. Proportion of samples, screened at DGA Status 1|2|3, assigned given diagnosis in accordance to RR.
\item{} \$\code{RR\_Ln1}: Optional: (\code{diagnosis} == 1|3): Numeric. Proportion of samples, screened at DGA Status 2 or 3, assigned given diagnosis in accordance to RR.
\item{} \$\code{IEC<n>\_L<n>}: Optional: (\code{diagnosis} == 1|4): Numeric. Proportion of samples, screened at DGA Status 1|2|3, assigned given diagnosis in accordance to IEC 1|2.
\item{} \$\code{IEC<n>\_Ln1}: Optional: (\code{diagnosis} == 1|4): Numeric. Proportion of samples, screened at DGA Status 2 or 3, assigned given diagnosis in accordance to IEC 1|2.

\end{itemize}


\end{itemize}

\end{Value}
%
\begin{Examples}
\begin{ExampleCode}
# Example 1: Two gases, two samples, individually assigned accuracies, no correlation, and no diagnostics
# gas_data: Dates and gas values (PPM). With only two samples, Table 4 will not be used.
gas_data_in <- data.frame(date = as.Date(c("2024-06-20","2024-06-24")),
                          H2 = c(8,4), CH4 = c(80,160))
# limit_data: IEEE Tables 1-4 Limits for each gas (PPM and PPM/Year).
limit_data_in <- data.frame(H2 = c(2, 3, 2, 0.05), CH4 = c(60,180,70,0.03))
# accuracy: Multiplied by gas value to give uncertainty.
# If "Normal", this represents 95% probability level, else 100%.
# Can also give a single value to be used for all gases
accuracy_in <- setNames(c(0.15, 0.50), c("H2", "CH4"))
# min_accuracy: Treated as PPM and used if the above would give an otherwise smaller value.
# Either can be given as NA to force the use of the other. This too can be given as single value.
min_accuracy_in = setNames(c(1, 5), c("H2", "CH4"))
C57104wMCM(gas_data = gas_data_in,
           limit_data = limit_data_in,
           accuracy = accuracy_in,
           min_accuracy = min_accuracy_in,
           distribution_shape = "N",
           rho = 0.5,
           diagnosis = 0,
           n = 1000,
           flex = FALSE, silent = FALSE, verbose = FALSE, validate = FALSE)

# Example 2: Seven gases, five samples, shared accuracies, specified correlation matrix, and Duval Triangle diagnostics
gas_data_in <- data.frame(date=as.Date(c("2024-06-20","2024-06-21","2024-06-22","2024-06-23","2024-06-24")),
                          H2=c(80,80,80,80,82), CH4=c(90,95,100,105,107), C2H6=c(80,100,80,80,82),
                          C2H4=c(30,30,31,29,31), C2H2=c(1,1,0,1,3), CO=c(800,850,860,850,852),
                          CO2=c(900,950,940,940,942))
limit_data_in <- data.frame(H2=c(80,200,40,20),CH4=c(90,150,30,10),C2H6=c(90,175,25,9),
                            C2H4=c(50,100,20,7),C2H2=c(1,2,0.5,0.5),CO=c(900,1100,20,100),
                            CO2=c(9000,12500,250,1000))
# rho: If given as a matrix, it is interpreted as a correlation matrix, else as rho.
# In this example, the correlation matrix is not positive definite.
# However, the flex argument allows for it to be adjusted to make it so.
rho_in <- matrix(rep(1,49), ncol=7)
C57104wMCM(gas_data = gas_data_in,
           limit_data = limit_data_in,
           accuracy = 0.15,
           min_accuracy = 1,
           distribution_shape = "T",
           rho = rho_in,
           diagnosis = "DT",
           n = 500,
           flex = TRUE, silent = FALSE, verbose = TRUE, validate = TRUE)
\end{ExampleCode}
\end{Examples}
\HeaderA{Define\_DTn}{Define\_IECn: Defines zones for plotting Duval Triangles 1, 4, or 5 Diagnostic Logic as per IEEE Table 6, D.3, and D.4, respectively.}{Define.Rul.DTn}
%
\begin{Description}
This defines zones for plotting in accordance to either IEEE C57.104-2019 Table 6 or Table D.3, or Table D.4. Table D.3 is modified here such that original definition for C: (CH4 >= 36 \& C2H6 >= 24) is instead: (CH4 >= 36 \& C2H6 < 24).
\end{Description}
%
\begin{Usage}
\begin{verbatim}
Define_DTn(DT_n)
\end{verbatim}
\end{Usage}
%
\begin{Arguments}
\begin{ldescription}
\item[\code{DT\_n}] String: Value corresponding to either Triangle 1, 4, or 5. Expected value: 1, 4, or 5.
\end{ldescription}
\end{Arguments}
%
\begin{Value}
List:
\begin{itemize}

\item{} \code{labels}: String. Used to define axes used in ternary plot.
\item{} \code{zones}: Numeric. Used for defining diagnostic zones used in ternary plot.
\item{} \code{colours}: RGB. Used for defining colours of zones.

\end{itemize}

\end{Value}
\HeaderA{Define\_IECn}{Define\_IECn: Defines zones for plotting IEC Table 1 or Table 2 Diagnostic Logic}{Define.Rul.IECn}
%
\begin{Description}
This defines zones for plotting in accordance to either IEC 60599:2022 Table 1 or Table 2. For Table 1, bounds shown in Figure B.1 are used.
\end{Description}
%
\begin{Usage}
\begin{verbatim}
Define_IECn(IEC_n)
\end{verbatim}
\end{Usage}
%
\begin{Arguments}
\begin{ldescription}
\item[\code{IEC\_n}] String: Value corresponding to either Table 1 or Table 2. Expected value: 1 or 2.
\end{ldescription}
\end{Arguments}
%
\begin{Value}
Data.frame:
\begin{itemize}

\item{} \code{xmin} | \code{xmax} | \code{ymin} | \code{ymax}: Numeric. Used to define rectangles in a ggplot.
\item{} \code{label}: String. Used for colour scaling in ggplot.
\item{} \code{name}: String. Used for defining which facet to plot in.

\end{itemize}

\end{Value}
\HeaderA{Define\_RR}{Define\_RR: Defines zones for plotting Rogers Ratio Diagnostic Logic as per IEEE Table 5}{Define.Rul.RR}
%
\begin{Description}
This defines zones for plotting Rogers Ratio in accordance to IEEE C57.104-2019 Table 5.
\end{Description}
%
\begin{Usage}
\begin{verbatim}
Define_RR()
\end{verbatim}
\end{Usage}
%
\begin{Value}
Data.frame:
\begin{itemize}

\item{} \code{xmin} | \code{xmax} | \code{ymin} | \code{ymax}: Numeric. Used to define rectangles in a ggplot.
\item{} \code{label}: String. Used for colour scaling in ggplot.
\item{} \code{name}: String. Used for defining which facet to plot in.

\end{itemize}

\end{Value}
\HeaderA{Diagnosis}{Diagnosis: Apply Diagnostics in accordance to either Duval Triangles, Rogers Ratios, or IEC Tables, or all.}{Diagnosis}
%
\begin{Description}
This function applies diagnostics in accordance to either Duval Triangles, Rogers Ratios, or IEC Tables, or all. The results are separated based on the IEEE Screening DGA Status Levels.
\end{Description}
%
\begin{Usage}
\begin{verbatim}
Diagnosis(
  y_1,
  gas_data,
  diagnosis,
  col_names,
  i_L1,
  i_L2,
  i_L3,
  silent = FALSE,
  verbose = FALSE
)
\end{verbatim}
\end{Usage}
%
\begin{Arguments}
\begin{ldescription}
\item[\code{y\_1}] Numeric Array: Values representing generated samples for the Monte Carlo based on the statistical description provided. Shaped: Gases x Samples.

\item[\code{gas\_data}] Data.frame: First column containing days since oldest sample. Rest of columns containing sample value in PPM.

\item[\code{diagnosis}] Data.frame: Columns containing the IEEE DGA Table limits for each gas. Each Table is on its own row; four rows expected. Caution: Table 4 is expressed as PPM/Day!

\item[\code{col\_names}] Strings: Gas Names. Order must match other named variables throughout. Expected a subset of <H2|CH4|C2H6|C2H4|C2H2|CO|CO2>.

\item[\code{i\_L1}] Logical Array. Mask indicated which generated samples were per-gas DGA Status Level 1. Length equal to n samples.

\item[\code{i\_L2}] Logical Array. Mask indicated which generated samples were per-gas DGA Status Level 2. Length equal to n samples.

\item[\code{i\_L3}] Logical Array. Mask indicated which generated samples were per-gas DGA Status Level 3. Length equal to n samples.

\item[\code{silent}] Boolean: Defaulted to FALSE. If TRUE heavily restricts outputs to console.

\item[\code{verbose}] Boolean: Defaulted to FALSE. If FALSE restricts outputs to console.
\end{ldescription}
\end{Arguments}
%
\begin{Value}
List containing a subset of the following:
\begin{itemize}

\item{} \$\code{DT<n>\_default\_data}: Optional: (\code{diagnosis} == 1|2): Numeric. Ratios relevant to DT 1|4|5 using default gas values.
\item{} \$\code{DT<n>\_0\_data}: Optional: (\code{diagnosis} == 1|2): Numeric Array. Ratios relevant to DT 1|4|5 using generated samples.
\item{} \$\code{DT<n>\_L<n>}: Optional: (\code{diagnosis} == 1|2): Numeric. Proportion of samples, screened at DGA Status 1|2|3, assigned given diagnosis in accordance to DT 1|4|5.
\item{} \$\code{DT<n>\_Ln1}: Optional: (\code{diagnosis} == 1|2): Numeric. Proportion of samples, screened at DGA Status 2 or 3, assigned given diagnosis in accordance to DT 1|4|5.
\item{} \$\code{gas\_L<n>\_default}: Optional: (\code{diagnosis} == 1|3|4): Numeric. Output for per-gas DGA Status Levels 1|2|3.
\item{} \$\code{Abs\_default\_default}: Optional: (\code{diagnosis} == 1|3|4): Numeric. Ratios relevant to RR and/or IEC 1|2 using default gas values.
\item{} \$\code{Abs\_0\_data}: Optional: (\code{diagnosis} == 1|3|4): Numeric Array. Ratios relevant to RR and/or IEC 1|2 using generated samples.
\item{} \$\code{RR\_L<n>}: Optional: (\code{diagnosis} == 1|3): Numeric. Proportion of samples, screened at DGA Status 1|2|3, assigned given diagnosis in accordance to RR.
\item{} \$\code{RR\_Ln1}: Optional: (\code{diagnosis} == 1|3): Numeric. Proportion of samples, screened at DGA Status 2 or 3, assigned given diagnosis in accordance to RR.
\item{} \$\code{IEC<n>\_L<n>}: Optional: (\code{diagnosis} == 1|4): Numeric. Proportion of samples, screened at DGA Status 1|2|3, assigned given diagnosis in accordance to IEC 1|2.
\item{} \$\code{IEC<n>\_Ln1}: Optional: (\code{diagnosis} == 1|4): Numeric. Proportion of samples, screened at DGA Status 2 or 3, assigned given diagnosis in accordance to IEC 1|2.

\end{itemize}

\end{Value}
%
\begin{Examples}
\begin{ExampleCode}
# y_1: Numeric array of generated samples, shaped: Gases, Samples
y_1_in <- array(c(82.39, 107.51,  82.39,  31.15,   3.03, 856.02, 946.45,  77.95, 101.71,  77.95,  29.47,   2.67, 809.89, 895.44,
                  78.98, 103.06,  78.98,  29.86,   2.75, 820.60, 907.28,  84.62, 110.42,  84.62,  31.99,   3.21, 879.26, 972.14,
                  74.19,  96.81,  74.19,  28.05,   2.37, 770.88, 852.31,  85.56, 111.64,  85.56,  32.35,   3.29, 888.97, 982.88,
                  84.71, 110.54,  84.71,  32.03,   3.22, 880.20, 973.18,  74.09,  96.68,  74.09,  28.01,   2.36, 769.80, 851.12,
                  83.24, 108.62,  83.24,  31.47,   3.10, 864.91, 956.27,  76.22,  99.45,  76.22,  28.81,   2.53, 791.90, 875.55), dim = c(7,10))
gas_data_in <- data.frame(date=c(4,3,2,1,0),
                          H2=c(80,80,80,80,82), CH4=c(90,95,100,105,107), C2H6=c(80,100,80,80,82),
                          C2H4=c(30,30,31,29,31), C2H2=c(1,1,0,1,3), CO=c(800,850,860,850,852),
                          CO2=c(900,950,940,940,942))
# temp_rndm_screen is just for generating screening mask as an example.
temp_rndm_screen <- sample(1:3, 10, replace=TRUE)
i_L1_in <- temp_rndm_screen == 1
i_L2_in <- temp_rndm_screen == 2
i_L3_in <- temp_rndm_screen == 3

Diagnosis(y_1 = y_1_in,
          gas_data = gas_data_in,
          diagnosis = 1,
          col_names = c("H2", "CH4", "C2H6", "C2H4", "C2H2", "CO", "CO2"),
          i_L1 = i_L1_in,
          i_L2 = i_L2_in,
          i_L3 = i_L3_in,
          silent = FALSE, verbose = FALSE)
\end{ExampleCode}
\end{Examples}
\HeaderA{Diagnosis\_Plot}{Diagnosis\_Plot: Plotting outputs from Diagnosis function for validation.}{Diagnosis.Rul.Plot}
%
\begin{Description}
This is internal function to plot generated diagnosis outputs for validation.
\end{Description}
%
\begin{Usage}
\begin{verbatim}
Diagnosis_Plot(DI, SC, PP)
\end{verbatim}
\end{Usage}
%
\begin{Arguments}
\begin{ldescription}
\item[\code{DI}] List: Outputs from Diagnosis function.

\item[\code{SC}] List: Outputs from Screening function.

\item[\code{PP}] List: Outputs from Preprocess function.
\end{ldescription}
\end{Arguments}
%
\begin{Examples}
\begin{ExampleCode}
# Not recreating required inputs to run this example
# Diagnosis_Plot(DI = diagnosed,
#                SC = screened,
#                PP = preprocessed)
\end{ExampleCode}
\end{Examples}
\HeaderA{Interpret\_DTn}{Interpret\_DTn: Applies Duval Triangles 1, 4, or 5 Diagnostic Logic as per IEEE Table 6, D.3, and D.4, respectively.}{Interpret.Rul.DTn}
%
\begin{Description}
This functions interprets the inputted gas sample ratios to determine the appropriate diagnosis for each in accordance to either IEEE C57.104-2019 Table 6 or Table D.3, or Table D.4. Table D.3 is modified here such that original definition for C: (CH4 >= 36 \& C2H6 >= 24) is instead: (CH4 >= 36 \& C2H6 < 24).
\end{Description}
%
\begin{Usage}
\begin{verbatim}
Interpret_DTn(DT_n, y_in, n_in)
\end{verbatim}
\end{Usage}
%
\begin{Arguments}
\begin{ldescription}
\item[\code{DT\_n}] String: Value corresponding to either Triangle 1, 4, or 5. Expected value: 1, 4, or 5.

\item[\code{y\_in}] Numeric Array: Values representing generated gas ratios, assumed order: DT1: C2H4, CH4, C2H2, DT4: CH4, H2, C2H6, DT5: C2H4, CH4, C2H6. Assumed shape: 3 x Samples.

\item[\code{n\_in}] Numeric: Number of samples to be diagnosed.
\end{ldescription}
\end{Arguments}
%
\begin{Value}
Strings: List of Diagnoses, one for each inputted sample.
\end{Value}
%
\begin{Examples}
\begin{ExampleCode}
Interpret_DTn(DT_n = "4",
               y_in = array(c(0.090, 0.002, 0.401, 0.120, 0.001, 0.310), dim = c(3,2)),
               n_in = 2)
\end{ExampleCode}
\end{Examples}
\HeaderA{Interpret\_IECn}{Interpret\_IECn: Applies IEC Table 1 or Table 2 Diagnostic Logic}{Interpret.Rul.IECn}
%
\begin{Description}
This functions interprets the inputted gas sample ratios to determine the appropriate diagnosis for each in accordance to either IEC 60599:2022 Table 1 or Table 2. For Table 1, bounds shown in Figure B.1 are used.
\end{Description}
%
\begin{Usage}
\begin{verbatim}
Interpret_IECn(IEC_n, y_in, n_in)
\end{verbatim}
\end{Usage}
%
\begin{Arguments}
\begin{ldescription}
\item[\code{IEC\_n}] String: Value corresponding to either Table 1 or Table 2. Expected value: 1 or 2.

\item[\code{y\_in}] Numeric Array: Values representing generated gas ratios, assumed order: C2H2/C2H4, CH4/H2, C2H4/C2H6, assumed shape: 3 x Samples.

\item[\code{n\_in}] Numeric: Number of samples to be diagnosed.
\end{ldescription}
\end{Arguments}
%
\begin{Value}
Strings: List of Diagnoses, one for each inputted sample.
\end{Value}
%
\begin{Examples}
\begin{ExampleCode}
Interpret_IECn(IEC_n = "1",
               y_in = array(c(0.090, 0.002, 0.401, 0.120, 0.001, 0.310), dim = c(3,2)),
               n_in = 2)
\end{ExampleCode}
\end{Examples}
\HeaderA{Interpret\_RR}{Interpret\_RR: Applies Rogers Ratio Diagnostic Logic as per IEEE Table 5}{Interpret.Rul.RR}
%
\begin{Description}
This functions interprets the inputted gas sample ratios to determine the appropriate diagnosis for each in accordance to IEEE C57.104-2019 Table 5.
\end{Description}
%
\begin{Usage}
\begin{verbatim}
Interpret_RR(y_in, n_in)
\end{verbatim}
\end{Usage}
%
\begin{Arguments}
\begin{ldescription}
\item[\code{y\_in}] Numeric Array: Values representing generated gas ratios, assumed order: C2H2/C2H4, CH4/H2, C2H4/C2H6, assumed shape: 3 x Samples.

\item[\code{n\_in}] Numeric: Number of samples to be diagnosed.
\end{ldescription}
\end{Arguments}
%
\begin{Value}
Strings: List of Diagnoses, one for each inputted sample.
\end{Value}
%
\begin{Examples}
\begin{ExampleCode}
Interpret_RR(y_in = array(c(0.090, 0.002, 0.401, 0.120, 0.001, 0.310), dim = c(3,2)),
             n_in = 2)
\end{ExampleCode}
\end{Examples}
\HeaderA{Preprocess}{Preprocess: Validate inputs and preprocess them into expected data formats / values}{Preprocess}
%
\begin{Description}
This function is intended to capture errors. No further validation is then done beyond this function.
\end{Description}
%
\begin{Usage}
\begin{verbatim}
Preprocess(
  gas_data,
  limit_data,
  accuracy,
  min_accuracy,
  distribution_shape,
  rho,
  diagnosis,
  n,
  flex = FALSE,
  silent = FALSE,
  verbose = FALSE,
  validate = FALSE
)
\end{verbatim}
\end{Usage}
%
\begin{Arguments}
\begin{ldescription}
\item[\code{gas\_data}] Data.frame: First column containing date of sample. Cannot have two samples on same day. Rest of columns containing sample value in PPM.

\item[\code{limit\_data}] Data.frame: Columns containing the IEEE DGA Table limits for each gas. Each Table is on its own row; four rows expected.

\item[\code{accuracy}] Named Numeric: Values to be multiplied by gas value to give uncertainty. Can either be same value for all gases, or individual values for each gas. Expected range: 0-1.

\item[\code{min\_accuracy}] Named Numeric: Absolute Values (PPM) used as minimum uncertainty. Can either be same value for all gases, or individual values for each gas.

\item[\code{distribution\_shape}] Numeric: Distribution shape representing uncertainty. "N" = Normal, "T" = Triangular, "U" = Uniform. If Normal, Accuracy is interpreted as 95\% confidence interval, else as 100\%.

\item[\code{rho}] Numeric: Value representing rho for correlations. Can either be same value for all gases, individual values for each gas, or a matrix intended to represent the \code{cor\_mat}.

\item[\code{diagnosis}] Mixed: Used to specify which diagnoses to conduct. 0 = None, 1 = All, "DT" = Duval Triangles 1|4|5 only, "RR" = Rogers Ratio Only, "IEC" = IEC Tables 1|2 Only.

\item[\code{n}] Numeric: Number of samples to generate for the Monte Carlo for each gas. If a value <1, it is assumed as an estimate for the error tolerance, then Wald Interval used to estimate n.

\item[\code{flex}] Boolean: Defaulted to FALSE. If TRUE, and \code{rho} is assumed a correlation matrix equivalent, it will attempt to force it to become a positive definite if needed.

\item[\code{silent}] Boolean: Defaulted to FALSE.

\item[\code{verbose}] Boolean: Defaulted to FALSE.

\item[\code{validate}] Boolean: Defaulted to FALSE.
\end{ldescription}
\end{Arguments}
%
\begin{Value}
List containing:
\begin{itemize}

\item{} \$\code{gas\_data}: Data.frame. First column as Date converted to days since oldest sample. Rest as gas values in PPM.
\item{} \$\code{limit\_data}: Data.frame. Unchanged from input.
\item{} \$\code{accuracy}: Numeric. Unchanged from input.
\item{} \$\code{min\_accuracy}: Numeric. Unchanged from input.
\item{} \$\code{distribution\_shape}: Numeric. 1 = Normal, 2 = Triangular, 3 = Uniform.
\item{} \$\code{rho}: Numeric or NA. NA if \code{cor\_mat} is being used.
\item{} \$\code{diagnosis}: Numeric. 0 = None, 1 = All, 2 = Duval Triangles, 3 = Rogers Ratios, 4 = IEC Tables.
\item{} \$\code{n}: Numeric. Number of samples used for Monte Carlo Simulation.
\item{} \$\code{col\_names}: Names of found gas names.
\item{} \$\code{cor\_mat}: Matrix or NA. NA if \code{rho} is being used.
\item{} \$\code{sd\_data}: Data.frame. Same as \code{gas\_data} but with the values representing standard deviation and not gas values.

\end{itemize}

\end{Value}
%
\begin{Examples}
\begin{ExampleCode}
Preprocess(gas_data = data.frame(date = as.Date(c("2024-06-20","2024-06-21","2024-06-22","2024-06-23","2024-06-24")), H2 = c(8,4,2,1,0.5), CH4 = c(10,20,40,80,160)),
           limit_data = data.frame(H2 = c(2, 3, 2, 0.05), CH4 = c(60,180,70,0.03)),
           accuracy = setNames(c(0.15, 0.50), c("H2", "CH4")),
           min_accuracy = setNames(c(1, 5), c("H2", "CH4")),
           distribution_shape = "N",
           rho = 0.5,
           diagnosis = 0,
           n = 1000,
           flex = FALSE, silent = FALSE, verbose = FALSE, validate = FALSE)
Preprocess(gas_data = data.frame(date=as.Date(c("2024-06-20","2024-06-21","2024-06-22","2024-06-23","2024-06-24")), H2=c(80,80,80,80,82),CH4=c(90,95,100,105,107),C2H6=c(80,100,80,80,82),C2H4=c(30,30,31,29,31),C2H2=c(1,1,0,1,3),CO=c(800,850,860,850,852),CO2=c(900,950,940,940,942)),
           limit_data = data.frame(H2=c(80,200,40,20),CH4=c(90,150,30,10),C2H6=c(90,175,25,9),C2H4=c(50,100,20,7),C2H2=c(1,2,0.5,0.5),CO=c(900,1100,20,100),CO2=c(9000,12500,250,1000)),
           accuracy = 0.15,
           min_accuracy = 1,
           distribution_shape = "T",
           rho = matrix(rep(1,49), ncol=7),
           diagnosis = "DT",
           n = 1000,
           flex = TRUE, silent = FALSE, verbose = TRUE, validate = TRUE)
\end{ExampleCode}
\end{Examples}
\HeaderA{Sample}{Sample: Randomly generate samples to later apply Screening to for Monte Carlo estimation}{Sample}
%
\begin{Description}
This function generates samples for the Monte Carlo based on the statistical description provided.
\end{Description}
%
\begin{Usage}
\begin{verbatim}
Sample(
  gas_data,
  limit_data,
  accuracy,
  min_accuracy,
  distribution_shape,
  rho,
  cor_mat,
  n,
  col_names,
  sd_data
)
\end{verbatim}
\end{Usage}
%
\begin{Arguments}
\begin{ldescription}
\item[\code{gas\_data}] Data.frame: First column containing days since oldest sample. Rest of columns containing sample value in PPM.

\item[\code{limit\_data}] Data.frame: Columns containing the IEEE DGA Table limits for each gas. Each Table is on its own row; four rows expected. Caution: Table 4 is expressed as PPM/Day!

\item[\code{accuracy}] Named Numeric: Values to be multiplied by gas value to give uncertainty. Can either be same value for all gases, or individual values for each gas. Expected range: 0-1.

\item[\code{min\_accuracy}] Named Numeric: Absolute Values (PPM) used as minimum uncertainty. Can either be same value for all gases, or individual values for each gas.

\item[\code{distribution\_shape}] Numeric: Distribution shape representing uncertainty. 1 = Normal, 2 = Triangular, 3 = Uniform. If Normal, Accuracy is interpreted as 95\% confidence interval, else as 100\%.

\item[\code{rho}] Numeric: Value representing rho for correlations. Can either be same value for all gases, or individual values for each gas. NA if using \code{cor\_mat} instead.

\item[\code{cor\_mat}] Numeric Matrix: Values representing correlation matrix. NA if using \code{rho} instead.

\item[\code{n}] Numeric: Number of samples to generate for the Monte Carlo for each gas.

\item[\code{col\_names}] Strings: Gas Names. Order must match other named variables throughout. Expected a subset of <H2|CH4|C2H6|C2H4|C2H2|CO|CO2>.

\item[\code{sd\_data}] Data.frame: First column containing days since oldest sample. Rest of columns containing standard deviation for sample value in PPM.
\end{ldescription}
\end{Arguments}
%
\begin{Value}
Numeric Array: Values representing generated samples for the Monte Carlo based on the statistical description provided. Shaped: Days x Gases x Samples.
\end{Value}
%
\begin{Examples}
\begin{ExampleCode}
Sample(gas_data = data.frame(date = c(4,3,2,1,0), H2 = c(8,4,2,1,0.5), CH4 = c(10,20,40,80,160)),
       limit_data = data.frame(H2 = c(2, 3, 2, 0.05), CH4 = c(60,180,70,0.03)),
       accuracy = setNames(c(0.15, 0.50), c("H2", "CH4")),
       min_accuracy = setNames(c(1, 5), c("H2", "CH4")),
       distribution_shape = 1,
       rho = 0.5,
       cor_mat = NA,
       n = 1000,
       col_names = c("H2", "CH4"),
       sd_data = data.frame(date = c(4,3,2,1,0), H2 = c(1.2,0.6,0.3,0.15,0.075), CH4 = c(5,10,20,40,80)))
Sample(gas_data = data.frame(date = c(4,3,2,1,0), H2 = c(8,4,2,1,0.5), CH4 = c(10,20,40,80,160)),
       limit_data = data.frame(H2 = c(2, 3, 2, 0.05), CH4 = c(60,180,70,0.03)),
       accuracy = 0.15,
       min_accuracy = 1,
       distribution_shape = 1,
       rho = NA,
       cor_mat = matrix(c(1,0,0,1), ncol=2),
       n = 1000,
       col_names = c("H2", "CH4"),
       sd_data = data.frame(date = c(4,3,2,1,0), H2 = c(1.2,0.6,0.3,0.15,0.075), CH4 = c(0.75,1.50,3,6,12)))
\end{ExampleCode}
\end{Examples}
\HeaderA{Sample\_Plot}{Sample\_Plot: Plotting outputs from Sample function for validation.}{Sample.Rul.Plot}
%
\begin{Description}
This is internal function to plot generated samples for validation.
\end{Description}
%
\begin{Usage}
\begin{verbatim}
Sample_Plot(y_N, PP)
\end{verbatim}
\end{Usage}
%
\begin{Arguments}
\begin{ldescription}
\item[\code{y\_N}] Numeric Array: Values representing generated samples for the Monte Carlo based on the statistical description provided. Shaped: Days x Gases x Samples.

\item[\code{PP}] List: Outputs from Preprocess function.
\end{ldescription}
\end{Arguments}
%
\begin{Examples}
\begin{ExampleCode}
# Not recreating required inputs to run this example
# Sample_Plot(y_1 = y_N[1,,],
#             PP = preprocessed)
\end{ExampleCode}
\end{Examples}
\HeaderA{Screening}{Screening: Apply Simplified IEEE C57.104-2019 Screening - Use all samples for Tables 1-4.}{Screening}
%
\begin{Description}
This function applies simplified IEEE C57.104-2019 Screening. It uses all samples for Tables 1-4 to give a per-gas and combined DGA Status. The frequency of given outcomes is interpreted as probabilities of them occurring.
\end{Description}
%
\begin{Usage}
\begin{verbatim}
Screening(y_N, limit_data, gas_data, silent = FALSE, verbose = FALSE)
\end{verbatim}
\end{Usage}
%
\begin{Arguments}
\begin{ldescription}
\item[\code{y\_N}] Numeric Array: Values representing generated samples for the Monte Carlo based on the statistical description provided. Shaped: Days x Gases x Samples.

\item[\code{limit\_data}] Data.frame: Columns containing the IEEE DGA Table limits for each gas. Each Table is on its own row; four rows expected. Caution: Table 4 is expressed as PPM/Day!

\item[\code{gas\_data}] Data.frame: First column containing days since oldest sample. Rest of columns containing sample value in PPM.

\item[\code{silent}] Boolean: Defaulted to FALSE.

\item[\code{verbose}] Boolean: Defaulted to FALSE.
\end{ldescription}
\end{Arguments}
%
\begin{Value}
List containing:
\begin{itemize}

\item{} \$\code{protocol}: Numeric. 1|2|3 depending on if 1, 2, or more samples.
\item{} \$\code{T<n>\_default}: Output for Table  using default gas values.
\item{} \$\code{T<n>}: Probability of output for Table  using generated samples.
\item{} \$\code{T<n>\_bounds}: Numeric. Associated estimate of confidence interval.
\item{} \$\code{gas\_L<n>\_default}: Numeric. Output for per-gas DGA Status Level .
\item{} \$\code{gas\_L<n>}: Numeric. Probability of output for per-gas DGA Status Level .
\item{} \$\code{gas\_L<n>\_bounds}: Numeric. Associated confidence interval.
\item{} \$\code{L<n>\_default}: Numeric. Output for combined DGA Status Level .
\item{} \$\code{L<n>}: Numeric. Probability of output for combined DGA Status Level .
\item{} \$\code{L<n>\_bounds}: Numeric. Mask indicated which generated samples were combined DGA Status Level .
\item{} \$\code{L<n>\_mask}: Logical Array. Associated estimate of confidence interval.
\item{} \$\code{T12\_vals\_default}: Numeric. Metric values using default gas values used for Tables 1 and 2.
\item{} \$\code{T3\_vals\_default}: Numeric. Metric values using default gas values used for Table 3.
\item{} \$\code{T4\_vals\_default}: Numeric. Metric values using default gas values used for Tables 4.
\item{} \$\code{T12\_vals}: Numeric. Metric values using generated samples used for Tables 1 and 2.
\item{} \$\code{T3\_vals}: Numeric. Metric values using generated samples used for Table 3.
\item{} \$\code{T4\_vals}: Numeric. Metric values using generated samples used for Tables 4.
\item{} \$\code{GL<n>\_mask}: Logical Array. Mask indicated which generated samples were per-gas DGA Status Level .

\end{itemize}

\end{Value}
%
\begin{Examples}
\begin{ExampleCode}
Screening(y_N = array(c(2,1,0.5,40,80,160,2.1,1.1,0.6,40.1,80.1,160.1), dim = c(2,2,2)),
          gas_data = data.frame(date = as.Date(c(2,1,0)), H2 = c(8,4,1), CH4 = c(40,80,160)),
          limit_data = data.frame(H2 = c(2, 3, 2, 0.05), CH4 = c(60,180,70,0.03)),
          silent = FALSE, verbose = FALSE)
\end{ExampleCode}
\end{Examples}
\HeaderA{Screening\_Plot}{Screening\_Plot: Plotting outputs from Screening function for validation.}{Screening.Rul.Plot}
%
\begin{Description}
This is internal function to plot generated screening outputs for validation.
\end{Description}
%
\begin{Usage}
\begin{verbatim}
Screening_Plot(SC, PP)
\end{verbatim}
\end{Usage}
%
\begin{Arguments}
\begin{ldescription}
\item[\code{SC}] List: Outputs from Screening function.

\item[\code{PP}] List: Outputs from Preprocess function.
\end{ldescription}
\end{Arguments}
%
\begin{Examples}
\begin{ExampleCode}
# Not recreating required inputs to run this example
# Screening_Plot(SC = screened,
#                PP = preprocessed)
\end{ExampleCode}
\end{Examples}
\printindex{}
\end{document}
